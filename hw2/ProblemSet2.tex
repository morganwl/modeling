\documentclass[12pt]{article}
%\usepackage{palatino, amsfonts, amsmath, fancyheadings}
\usepackage{amsfonts, amsmath, fancyhdr,lastpage,multirow}
\usepackage{amsthm}
\usepackage{graphicx}
\usepackage{listings}
\pagestyle{fancy}

\usepackage[noend]{algpseudocode}
\usepackage{algorithm}

%% For marking tables

\usepackage{tikz}
\usetikzlibrary{fit,shapes.misc}

\newcommand\marktopleft[1]{%
    \tikz[overlay,remember picture] 
        \node (marker-#1-a) at (0,1.5ex) {};%
}
\newcommand\markbottomright[1]{%
    \tikz[overlay,remember picture] 
        \node (marker-#1-b) at (0,0) {};%
    \tikz[overlay,remember picture,thick,dashed,inner sep=3pt]
        \node[draw, rectangle,fit=(marker-#1-a.center) (marker-#1-b.center)] {};%
}
%% End marking tables


\topmargin = -0.8 in
\oddsidemargin = 0.0 in
\evensidemargin = 0.0 in
\textheight = 9 in
\textwidth = 6.75 in

\newtheorem{theorem}{Theorem}   
\newtheorem{result}{Result}
\newtheorem{definition}{Defintion}  
\newcounter{problem}
\newcounter{solution}
\def\bibite{$\div \hskip-0.77em \times$}
\def\endex{\bibite \kern-3pt \bibite \kern-3pt \bibite}
\def\theproblem{\arabic{problem}}
\newenvironment{problem}
     {\refstepcounter{problem}%
    \medskip\noindent{\bfseries Problem} {\bfseries \theproblem}
      }
%   {\vskip-1.5em  \hphantom{a} \hfill \endex
%     \smallskip}

\newenvironment{solution}
     { 
         % \refstepcounter{solution}%
         \setlength{\parskip}{0.5em}
         \setlength{\parindent}{0em}
         \setlength{\leftskip}{1em}
         \medskip\noindent{\bfseries Solution to Problem {\theproblem}:}
     }
     {
%     \vskip-1.5em  \hphantom{a} \hfill \endex
      \medskip
       }
\newcommand{\res}[1]{Result~\ref{res:#1}}
\newcommand{\eq}[1]{(\ref{eq:#1})}
\newcommand{\fig}[1]{Figure~\ref{fig:#1}}
\newcommand{\alg}[1]{Algorithm \ref{alg:#1}}
\renewcommand{\labelenumi}{(\alph{enumi})}

% Felisa's Shorthand notations
\def\points#1{{\bfseries [{#1} points]}\\%
}
\def\marks#1{{\bfseries [{#1} points]}}
\font\cmss=cmss10 scaled 1100
\font\goth=eufm10 at 11pt
\def\ds{\displaystyle}
\def\Field{{\hbox{\goth F}}}
\def\Filtration{{\mathbb F}}
\def\borel{{\cal B}}
\def\Real{{\mathbb R}}   % Real numbers
\def\Int{\mathbb N}  % Integer numbers
\def\Prob{{\mathbb P}}
\def\Esp{{\mathbb E}}
\def\prob{\hbox{\cmss P}}
\def\esp{\hbox{\cmss E}}   
\def\var{\hbox{\cmss Var}}   
\def\cov{\hbox{\cmss Cov}}   
\def\eqdist{{\,\buildrel \cal L\over = }\,}    
\def\todist{{\,\buildrel \cal L\over \Longrightarrow }\,}    
\def\eqdef{{\,\buildrel {\rm def} \over = }\,}
%
\def\normal{{\cal N}}  
\def\th{{\theta}}
\def\Th{{\Theta}}
\def\la{{\lambda}}
\def\ind#1{{\mathbf 1}_{\{#1\}}}
\def\given {\,|\,}
\def\order{{\cal O}}
\def\GG{{\cal G}}
\def\loss{{\cal L}}
\def\eff{{\cal E}}
\def\euro{ \hbox{\small{\hskip-.3em$\subset$ \hskip - 1.15 em  $-$}}}
\def\ep{{\epsilon}}
\newcommand\pfrac[2]{{\left(\frac{#1}{#2}\right)}}

\lstnewenvironment{python}
{%
    % \mbox{}
    % \vspace*{\baselineskip}
    \lstset{
        xleftmargin=.5\leftmargin,
        linewidth=\linewidth,
        language=python
    }%
}
{}

\author{Morgan Wajda-Levie}
\date{October 20, 2022}



\begin{document}\sf

\lhead{ } \chead{ } \rhead{ } \lfoot{ } \cfoot{\sf -- page
\thepage~  of  \pageref{LastPage}  --  } \rfoot{ }                                                     %            <=====   PAGE NUMBERING

\title{\bfseries Computer Modeling  and  Simulation}

\maketitle

%\null \vskip - 5 cm

\begin{center}
\vskip - 1.5em { {\bfseries Problem Set 2}}
\end{center}

\begin{problem} \points{10}
Let  $\{X_n\}$ be a finite Markov Chain with state space $S$, and let $M=|S|<\infty$ be the number of states. Prove that if state $j$ is accessible from state $i$, then it is possible to attain state $j$ in at most $M$ time steps, starting at $X_0=i$. 
\end{problem}

\begin{solution}

    Assume that there exists some states $i,j$ where $j$ is accessible
    from $i$ and the minimum number of steps to reach $j$ from $i$ is
    $N$, where $N > |S|$.

    By the Markov property, this means that there exists some $X_0,
    X_1,...,X_T$ of length $N$ such that $X_0 = i, X_T=j$ and $\Prob(X_n
    \mid X{n-1}) > 0$ and $\Prob(X_n \mid X_m) = 0$ for all $m < n - 1$,
    because, if there exists a path from $X_m$ to $X_n$, then $X_T$ can
    be reached without passing through $X_{m+1},\ldots,X_{n-1}$,
    resulting in a sequence of length $<N$.

    However, if $N > |S|$, there must exist some distinct $X_n,X_m$ such
    that $X_n=X_m$, meaning that $\Prob(X_n \mid X_{m-1}) > 0$ and $m -
    1 < n -1$, which contradicts the Markov property.

    \raggedleft $\square$

\end{solution}


\begin{problem} \points{10}
Consider the  discrete queue model where at most a single customer arrives during a single service period, and the service time of a customer is a random variable $Z$ with the geometric probability distribution:

\[
\Prob(Z=k) = \alpha\, (1-\alpha)^{k-1}, \quad k=1,2,\ldots.
\]
Specify the transition probabilities of the Markov Chain whose state is the number of customers waiting for service or being served at the start of each period. Assume that the probability that a customer arrives in a period is $0<\beta<1$ and that no customer arrives with probability $1-\beta$.  
\end{problem}

\begin{solution}
    % type A: dominant
    % type a: recessive
    % p = probability of AA in population
    % q = probability of aa in population
    % r = probability of Aa in population

    Let $d$ be the probability that a single parent with a dominant
    phenotype will pass a recessive gene to their children. Because a
    parent with a recessive phenotype will pass a recessive gene on to
    their child w.p.1., $S_{10} = d$. Since $S_{11} = d * d$, we can
    show that $S_{11} = S_{10}$ without even calculating the probability
    for $d$.

    Because $AA$ is unable to transmit a recessive gene, and assuming
    the selection of genes to be transmitted is uniform, $d$ can be
    calculated as the probability of transmitting $a$ given $Aa$ times
    the probability of $Aa$ given $AA$ or $Aa$,
    $d = \frac{r}{2(p + r)}$.

    \begin{align*}
        \Prob(S_{11}) &= \left(\frac{r}{2(p+r)}\right)^2 \\
        \Prob(S_{10}) &= \frac{r}{2(p+r)}
    \end{align*}

\end{solution}


\begin{problem} \points{40}
This is a simulation of a typical station of public shared vehicles (such as the Citibike). The occupancy of the station at time $t$, denoted $X(t)$ is the number of vehicles parked. In this example we assume that all vehicles are ready to use (no need for repairs). Let $T=120$. During the interval of time $(0,T]$ the process evolves as follows, starting with $X(0)=10$. Bikes arrive at the station following a Poisson process with rate $\la = 6$. There are three types of clients that arrive requesting a vehicle according to independent Poisson process with respective rates $\mu_i, i=1,2,3$. Classes 1 and 2 are clients that pay annual memberships, and the revenue from these clients during $[0,T)$ is the aggregated (deterministic) amount $(K_1 \mu_1+ K_2\mu_2)T$, where $K_1 = .5, K_2=.1$. Class 3 describes clients that pay per ride an amount $K_3 = 1.25$. When annual members arrive at an empty station they are dissatisfied. This costs the system a penalty of $c_i, i=1,2$ where $c_1 = 1.00, c_2=0.25$. No penalty is paid for class 3 dissatisfactions. In your simulation, you will estimate the net profit during the interval of time $[0,T)$ using $\mu_1=3, \mu_2=1,\mu_3=4$.  
\begin{enumerate}
\item Describe the DES model and perform your simulations. Clarify how many replications of the interval $[0,T)$ you choose to use and how do you use these simulations in order to estimate the net profit. 
\item Now consider a retrospective model for the simulations instead. Use the following approach:
\begin{itemize}
\item[i)] Merge all Poisson processes. Name what theorems you use and what results from this merging. Describe the distribution of $M$, the total number of events in $[0,T)$. Given $M$, explain how you can calculate the times $(A_i, i=1,\ldots,M)$ of the $M$ events and name what theorems you use here. 
\item[ii)] Initialize your simulation generating $M$. Explain what method you use for this generation. 
\item[iii)] The rest of your code loops from $i=1$ to $M$. Explain how you calculate what type of event happens at every step in the code and name the theorems you use to justify this answer.
\end{itemize}
Clarify how many replications of the interval $[0,T)$ you choose to use and how do you use these simulations in order to estimate the net profit. 
\item Include a discussion comparing the two simulation approaches. 
\end{enumerate}
\end{problem}

\begin{solution}
    \begin{enumerate}
        \item
            Because the state following any state $j$ is always chosen
            from states in $\{0,1,\ldots,j-1\}$, we can see that this
            process will always monotonically decrease towards $0$. We
            can see that, if a given state $k$ is stated as a fraction
            of the state $j$ that proceeds it, that
            \[
                m \prod_{n=0}^t \frac{k}{X_n}
            \]
            will shrink towards $0$ as $t$ approaches $\tau$, where
            $\tau$ is the total number of steps until the process is
            absorbed in state $0$.

            We can calculate an expected value of $k$, given $j$, by
            summing the possible values of $k$, multiplied by their
            respective probability, given $j$.
            \begin{align*}
                \frac{2}{j^2} \sum_{k=1}^{j-1}k^2
                &= \frac{2(j-1)(j)(2j)}{6j^2}\\
                &= \frac{2j^3-2j^2}{3j^2}\\
                &= \frac{2j-2}{3} \\
                \mathbb E[k \mid j] &= \frac{2j-2}{3}
            \end{align*}

            Substituting $k$ for its expected value, we can see that,
            given sufficiently large $m$:

            \begin{align*}
                1 &= \left\lceil m\prod_{n=0}^\tau \frac{2 X_n-2}{3 X_n}
                \right\rceil\\
                  & \approx  m \prod_{n=0}^\tau \frac{2}{3}\\
                  &= m \left(\frac{2}{3}\right)^\tau\\
                  \text{such that}\\
                    \mathbb E[\tau] &=2\log_3m
            \end{align*}

            Because $O()$ represents an upper bound, this value is also
            $O(log_2m)$.

        \item

            The simulation starts with a value $m$ (the initial state),
            and continuously samples a new value $k$ based on the
            current state, until $k = 0$. To generate the random
            variable $k$ according to the given probabilities, a reverse
            linear search is used, starting with the largest and most
            probable value of $k$, and decreasing $k$ until the value
            according to $1 - F(k)$. Moving backwards improves
            performance over a forward search by a factor of $\approx
            2$.

            In order to guarantee no more than a 5\% error rate, I
            performed the simulation in batches, gathering the mean
            simulated value and the sample variance. After each batch,
            or \emph{replication}, I calculated the running average for
            the variance, and divided this by the total simulations.
            Once this value was within my desired confidence ($<0.05$),
            I returned the observed mean and variance.

            While I could not calculate the mean iterations required for
            an arbitrary $m$, I tested a number of large values of $m$,
            and found that the theoretical minus the observed was always
            less than 1 (and greater than 0) and generally less than
            0.5, so I would say that this is reasonably consistent.

            \newpage
            \begin{python}

def sample_k(j):
    """Generate a value for k, given j, according to specified
    probabilities."""
    u = random()
    k = j - 1
    cdf = 2 * k / j**2
    while k > 0:
        if u <= cdf:
            return k
        k -= 1
        cdf += 2 * k / j**2
    return 0
            \end{python}

    \end{enumerate}
\end{solution}


That's time! I'd love to have the time to work further, but I have to
honor the cut-off.

\end{document}
