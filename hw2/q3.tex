\begin{solution}

    \begin{enumerate}
        \item The DES model defines 4 events, one for bike arrivals, and
            one for each class of rider arrivals. Events are defined
            with an arrival rate and an effect on the current process
            state when an event of that type is triggered.

            A priority queue is initialized with one event of each type,
            sorted by arrival time, where arrival times are sampled
            randomly according to a poisson process with the respective
            rates for each type of event.

            A loop pops the soonest event off this queue, exiting if
            that event occurs after the end of the simulation interval.
            Otherwise, the event is triggered, updating the state
            accordingly, and then a new event of the same type is
            created, with a random arrival time, and added to the event
            queue.

            Once the first event after the simulation interval has been
            reached (but before it has been triggered), the profit is
            calculated.

            I performed this simulation 5000 times, because that seemed
            to converge into a fairly consistent result, and observed a
            mean profit of 55 units.

            \begin{python}
class Event:
    """An event, initialized with a random arrival time,
    according to a poisson process and specified rate,
    and offset from the current time."""
    rate = 0
    def __init__(self, time):
        self.time = time + poisson(self.rate)

    def trigger(self, state):
        """Triggers the event and modifies the current state
        accordingly."""
        # defined for each subclass

    def get_next(self):
        """Create a new event of the same type,
        occuring some time in the future based on
        that event's inter-arrival time distribution."""
        return type(self)(self.time)
            \end{python}

            \begin{python}
def simulate(interval=T, verbose=False):
    """Perform one simulation and return the final profit."""
    t = 0
    state = {'num_bikes': INITIAL_BIKES, 'revenue': 0, 'cost': 0}

    future_events = PriorityQueue()
    for et in event_types:
        future_events.put(et(t))

    while True:
        event = future_events.get()
        if event.time > interval:
            break
        t = event.time
        event.trigger(state)
        if verbose:
            print_event(t, event, state)
        future_events.put(event.get_next())
    return profit(interval, state)
            \end{python}

        \item

            \textbf{scratch most of the below, I think we are better off
                modeling M as a binomial variable, because the value of
            $e^{-\lambda\tau} < \epsilon$ when $\tau=120$.}

            Because the four potential events are all independent and
            disjoint, the probability of any one event is equal to the
            sum of the probabilities of each type of event. (Some might
            call this the addition rule of probability.) The resulting
            process is a poisson process with $\lambda = \sum_{i=0}^4
            \lambda_i$.

            Because the PMF of a Poisson is not invertible, we cannot
            sample M by applying the inverse function to a uniform
            random variable, as we did for generating random arrival
            times. A simple, though maybe not the most efficient,
            approach is to generate a table of the CDF, $F(.)$ and
            select the value M for which some $x \sim U \in [F(M),
            F(M+1))$.

            I'm not sure exactly how I would calculate the times $A_i$,
            such that $A_M$ is guaranteed to be within the given
            interval. That is, I could sample the inter-arrival times of
            the poisson distribution, just as I did in the previous
            version, but those random inter-arrival times could result
            in sum that is greater than T. That said, I'm not sure that
            I need to know the times $A_i$ for the sake of this
            simulation, because the number of bikes present at the
            station is not dependent on inter-arrival times.

            I can identify my events by noting that
            \[
                \Prob(\xi_1 \mid \{\xi_1,...,\xi_4\}) = \frac{\Prob(\xi_1)}
                {\Prob(\xi_1,...,\xi_4)}
            \]
            and create a CDF similarly to how I did for the value of M.

        \item \textbf{TO-DO: And this}
    \end{enumerate}

\end{solution}
