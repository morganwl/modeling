\documentclass[12pt]{article}

\usepackage{modelset}

\title{\bfseries Computer Modeling  and  Simulation}
\author{Morgan Wajda-Levie}
\date{December 4, 2022}

\begin{document}\sf

\lhead{ } \chead{ } \rhead{ } \lfoot{ } \cfoot{\sf -- page
\thepage~  of  \pageref{LastPage}  --  } \rfoot{ }                                                     %            <=====   PAGE NUMBERING

\maketitle

%\null \vskip - 5 cm

\begin{center}
\vskip - 1.5 em { {\bfseries Problem Set 4}}
\end{center}

\begin{problem}\marks{15}
The occupancy of a public bike station at time $t$, denoted $X(t)$ is
the number of vehicles parked. In this example we assume that all
vehicles are ready to use (no need for repairs). Let $T=120$. During the
interval of time $(0,T]$ the process evolves as follows, starting with
$X(0)=10$. Bikes arrive at the station following a Poisson process with
rate $\la = 6$. There are three types of clients that arrive requesting
a vehicle according to independent Poisson process with respective rates
$\mu_i, i=1,2,3$. Classes 1 and 2 are clients that pay annual
memberships, and the revenue from these clients during $[0,T)$ is the
aggregated (deterministic) amount $(K_1 \mu_1+ K_2\mu_2)T$, where $K_1 =
.5, K_2=.1$. Class 3 describes clients that pay per ride an amount $K_3
= 1.25$. When annual members arrive at an empty station they are
dissatisfied and leave (either they  walk away  or they seek other means
of transportation). Each dissatisfied client costs the system a penalty
of $c_i, i=1,2$ where $c_1 = 1.00, c_2=0.25$. No penalty is paid for
class 3 dissatisfactions. In your simulation, you will estimate the net
profit during the interval of time $[0,T)$ using $\mu_1=3,
\mu_2=1,\mu_3=4$. 
\begin{enumerate}
\item Develop the formulas to estimate only the cost of dissatisfaction
    (total penalty costs) using a {\bfseries conditioning argument for
    variance reduction}. [{\it Hint: } Given an interval of time
    $(t_1,t_2]$ the expected number of Poisson arrivals at rate $\bar
    \mu$ is $\bar \mu (t_2-t_1)$.] 
\item Implement the resulting formula to run your simulations both in
    DES and retrospective simulations in order to estimate the total
    profit.  ({\it You should be able to re-use the code from Problem
    Set 2}).
\item Provide confidence intervals and compare the methods (simulation
    efficiency) with and without the variance reduction method. 
\item {\bfseries Bonus points:} Discuss how would you implement a
    similar conditioning argument to evaluate the revenue from the
    single ride clients of type 3.  
\end{enumerate}

\end{problem}

\begin{solution}

    Assume that there exists some states $i,j$ where $j$ is accessible
    from $i$ and the minimum number of steps to reach $j$ from $i$ is
    $N$, where $N > |S|$.

    By the Markov property, this means that there exists some $X_0,
    X_1,...,X_T$ of length $N$ such that $X_0 = i, X_T=j$ and $\Prob(X_n
    \mid X{n-1}) > 0$ and $\Prob(X_n \mid X_m) = 0$ for all $m < n - 1$,
    because, if there exists a path from $X_m$ to $X_n$, then $X_T$ can
    be reached without passing through $X_{m+1},\ldots,X_{n-1}$,
    resulting in a sequence of length $<N$.

    However, if $N > |S|$, there must exist some distinct $X_n,X_m$ such
    that $X_n=X_m$, meaning that $\Prob(X_n \mid X_{m-1}) > 0$ and $m -
    1 < n -1$, which contradicts the Markov property.

    \raggedleft $\square$

\end{solution}


\begin{problem}\marks{50}  Consider a M/GI/1 queue with Poisson arrivals
    of rate $\la$ and iid service times $\{S_i\}$ with  distribution
    $\Gamma(3,0.25)$. The goal is to estimate the stationary average
    queue length $\th$. Consider the Petri-Net model for the simulation.
    ({\it You should be able to re-use your code from Problem Set 2}). 

\begin{enumerate}
\item  Let $\alpha=0.05$. Use an adaptive algorithm to stop the
    simulation so that the approximate confidence interval has precision
    $\ep = 0.1$. Assume that you do not know the mean and variance of
    the service distribution (that is, your program could run by reading
    consecutive service times from a file with historical data, for
    example). Explain your choice of the algorithm to estimate the
    confidence interval (independent runs, batch means, discarding of
    ``warm-up'' period, etc). 
\item Show that the total number of iterations in your simulation model
    using the stopping rule that you have defined is a random stopping
    time with respect to the simulation process. 
\item Perform 20 independent simulation runs to estimate the coverage
    probability for $\th$. Discuss your results.
 %\item Discuss the results of the two different approaches: DES {\it versus} Petri Nets. 
\item Explain how to apply each of the following methods for variance
    reduction for the problem above when implementing the Petri net
    model for simulation:
\begin{itemize}
\item Antithetic random variables.
\item Control variable (what variable do you propose to use for the control?)
\end{itemize}
\item Perform the simulations with the added methods using the same
    stopping criterion and record the CPU times for the 20 replications.
    Compare your results with and without variance reduction and
    discuss. 
\end{enumerate}
\end{problem}





\begin{problem} \marks{10}
There are $N$ individuals in a population, some of whom have a certain
viral infection that spreads as follows. Contacts between two members of
this population occur in accordance with a Poisson process of rate
$\la$. When a contact occurs, it is equally likely to involve any of the
$N\choose2$ pairs of individuals. If a contact involves an infected and
a healthy individual, then with probability $p$ the non-infected one
becomes infected. Once infected, an individual remains ill for an
exponential amount of time with intensity $\mu$, after which she/he
becomes healthy again. We assume no deaths occur, and there is no
spontaneous infection. Let $X(t)$ denote the number of infected
individuals at time $t$. 
Discuss the application of Importance Sampling for variance reduction
and perform simulations ({\it You should be able to re-use your code
from Problem Set 3}). 
\end{problem}

\begin{problem} \marks{25}
You have decided to do consultation for modeling, simulation and
optimization. You offer various research  services: modeling,
statistical analysis, optimization, software development, etc. There are
$N$ such research stages (or ``tasks'') and they always follow a
specific order: research of type  $n$ is always followed  by that of
type $n-1$, for $N\ge n > 1$. The time (in hours) required to complete
stage $n$ follows a {\bfseries lognormal} distribution $F_n$ of mean
$\mu_n$. Potential clients arrive according to a Poisson process of rate
$\la$, and you take the job only if you are free at the time of arrival
of the client. If you are already working on a problem then you do not
take new contracts. Each problem starts at stage $n$ with probability
$p_n$ and ends at stage $1$, following all  intermediate stages. Let
$X(t)=n$ if at time $t$ you are working on stage $n$ of a problem, and
use $X(t)=0$ if at time $t$ you are free, waiting for new contracts. You
will charge $c$ dollars per hour of work. Assume that the probability
that clients accept your conditions is given by $P_c=(K-c)/K, 0\le c
<K$. If $c\ge K$ then $P_c=0$ (no client will hire you). For your
simulation, use $K=2, \la=2$, $N=5$ and $F_n\sim$ LN$(\th_n,\sigma^2)$,
with $\sigma^2=1$. Use $\mu= (0.3,  0.1, 0.2, 0.3, 0.1)^\top$. 

\begin{enumerate}
\item  \marks{5} Find the appropriate values of $\th_n$ so that the
    corresponding mean is $\mu_n$. 
\item \marks{10} Describe how to use CRN to produce simulations for sets
    of values $(c_1,\ldots, c_k)$. For this problem, under which
    conditions on your simulation will the CRN ensure variance reduction
    of the pairwise comparisons? 
\item  \marks{10} Implement the simulation for various values of $c$ to
    estimate the optimal one.  {\it You may re-use the code from Problem
    Set 3.} 
\item {\bfseries Bonus points:} Implement the Golden search method for
    finding the optimal value $c^*$.What precision criterion are you
    using for each of your pair-wise comparisons? 

\end{enumerate}

\end{problem}

\end{document}
