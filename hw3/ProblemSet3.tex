\documentclass[12pt]{article}

\usepackage{../modelset}


\title{\bfseries Computer Modeling  and  Simulation}
\author{Morgan Wajda-Levie}
\date{November 6, 2022}


\begin{document}\sf

\lhead{ } \chead{ } \rhead{ } \lfoot{ } \cfoot{\sf -- page
\thepage~  of  \pageref{LastPage}  --  } \rfoot{ }                                                     %            <=====   PAGE NUMBERING

\maketitle

%\null \vskip - 5 cm

\begin{center}
\vskip - 1.5 em { {\bfseries Problem Set 3}}
\end{center}

\begin{problem}\marks{15}
    Let $P^{(i)}$ be the transition matrix of an ergodic Markov chain
    with limit probabilities $\pi^{(i)}$, for $i=1,2$. 
    \begin{enumerate}
        \item \marks{5} Let $X_0=1$ be given. If the result of a coin
            toss is Heads (H) then $\{X_n\}$ will follow the dynamics
            given by $P^{(1)}$. Otherwise it will follow those given by
            $P^{(2)}$. Is $\{X_n\}$ a Markov chain? If so, determine the
            transition probabilities. Letting  $p=\Prob(H)$, calculate
            the limiting probabilities: $\lim_{n\to\infty}
            \Prob(X_n=i)$?

        \item \marks{10} Now suppose that at each step we first throw a
            coin. If it results $H$ then the following state is chosen
            according to $P^{(1)}$, otherwise it is chosen according to
            $P^{(2)}$. Is $\{X_n\}$ a Markov chain? If so, determine the
            transition probabilities. Show  by counterexample that the
            limit probabilities are not the same as in (a) above.
    \end{enumerate}
\end{problem}

\begin{solution}

    Assume that there exists some states $i,j$ where $j$ is accessible
    from $i$ and the minimum number of steps to reach $j$ from $i$ is
    $N$, where $N > |S|$.

    By the Markov property, this means that there exists some $X_0,
    X_1,...,X_T$ of length $N$ such that $X_0 = i, X_T=j$ and $\Prob(X_n
    \mid X{n-1}) > 0$ and $\Prob(X_n \mid X_m) = 0$ for all $m < n - 1$,
    because, if there exists a path from $X_m$ to $X_n$, then $X_T$ can
    be reached without passing through $X_{m+1},\ldots,X_{n-1}$,
    resulting in a sequence of length $<N$.

    However, if $N > |S|$, there must exist some distinct $X_n,X_m$ such
    that $X_n=X_m$, meaning that $\Prob(X_n \mid X_{m-1}) > 0$ and $m -
    1 < n -1$, which contradicts the Markov property.

    \raggedleft $\square$

\end{solution}


\begin{problem} \marks{10}
Consider a population of individuals each of whom possesses two genes
that can be either type $A$ or type $a$. Suppose that in outward
appearance (the {\em phenotype}) type $A$ is dominant and type $a$
recessive. Suppose that the population has stabilized, and the
percentages of individuals having respective gene pairs $AA$, $aa$ and
$Aa$ are $p,q,$ and $r$ Call an individual dominant or recessive
depending on phenotype. Let $S_{11}$ denote the probability that an
offspring of two dominant parents will be recessive, $S_{10}$ the
probability that the offspring of one recessive and one dominant parent
will be recessive. Compute $S_{11}$ and $S_{10}$ to show that $S_{11}=
S_{10}^2$. These quantities are known in genetics as {\em Snyder's
ratios}. 
\end{problem}

\begin{solution}
    % type A: dominant
    % type a: recessive
    % p = probability of AA in population
    % q = probability of aa in population
    % r = probability of Aa in population

    Let $d$ be the probability that a single parent with a dominant
    phenotype will pass a recessive gene to their children. Because a
    parent with a recessive phenotype will pass a recessive gene on to
    their child w.p.1., $S_{10} = d$. Since $S_{11} = d * d$, we can
    show that $S_{11} = S_{10}$ without even calculating the probability
    for $d$.

    Because $AA$ is unable to transmit a recessive gene, and assuming
    the selection of genes to be transmitted is uniform, $d$ can be
    calculated as the probability of transmitting $a$ given $Aa$ times
    the probability of $Aa$ given $AA$ or $Aa$,
    $d = \frac{r}{2(p + r)}$.

    \begin{align*}
        \Prob(S_{11}) &= \left(\frac{r}{2(p+r)}\right)^2 \\
        \Prob(S_{10}) &= \frac{r}{2(p+r)}
    \end{align*}

\end{solution}


\begin{problem}\marks{20}
An algorithm is built to find the zeroes of a function. If this
algorithm is in state $j$ at the $n$-th step, then the probability of
finding a zero in the next step is $1/j$. Otherwise its state will be $k
\in \{1,2,\ldots, j-1\}$ with probability $2k^/j^2$. 
\begin{enumerate}
\item \marks{10} Find the mean number of iterations of the algorithm
    when we start at state $m$ and show that this is $O(\log_2 m)$. 
\item \marks{10} Simulate this process as a Markov chain and estimate
    the number of iterations until the algorithm finds the zero. Explain
    how you write the code  in order to achieve a  relative error of
    $5\%$ in your final estimate.  Verify that your simulation results
    are consistent with your theoretical answer above. 
\end{enumerate}
\end{problem}

\begin{solution}

    \begin{enumerate}
        \item The DES model defines 4 events, one for bike arrivals, and
            one for each class of rider arrivals. Events are defined
            with an arrival rate and an effect on the current process
            state when an event of that type is triggered.

            A priority queue is initialized with one event of each type,
            sorted by arrival time, where arrival times are sampled
            randomly according to a poisson process with the respective
            rates for each type of event.

            A loop pops the soonest event off this queue, exiting if
            that event occurs after the end of the simulation interval.
            Otherwise, the event is triggered, updating the state
            accordingly, and then a new event of the same type is
            created, with a random arrival time, and added to the event
            queue.

            Once the first event after the simulation interval has been
            reached (but before it has been triggered), the profit is
            calculated.

            I performed this simulation 5000 times, because that seemed
            to converge into a fairly consistent result, and observed a
            mean profit of 55 units.

            \begin{python}
class Event:
    """An event, initialized with a random arrival time,
    according to a poisson process and specified rate,
    and offset from the current time."""
    rate = 0
    def __init__(self, time):
        self.time = time + poisson(self.rate)

    def trigger(self, state):
        """Triggers the event and modifies the current state
        accordingly."""
        # defined for each subclass

    def get_next(self):
        """Create a new event of the same type,
        occuring some time in the future based on
        that event's inter-arrival time distribution."""
        return type(self)(self.time)
            \end{python}

            \begin{python}
def simulate(interval=T, verbose=False):
    """Perform one simulation and return the final profit."""
    t = 0
    state = {'num_bikes': INITIAL_BIKES, 'revenue': 0, 'cost': 0}

    future_events = PriorityQueue()
    for et in event_types:
        future_events.put(et(t))

    while True:
        event = future_events.get()
        if event.time > interval:
            break
        t = event.time
        event.trigger(state)
        if verbose:
            print_event(t, event, state)
        future_events.put(event.get_next())
    return profit(interval, state)
            \end{python}

        \item

            \textbf{scratch most of the below, I think we are better off
                modeling M as a binomial variable, because the value of
            $e^{-\lambda\tau} < \epsilon$ when $\tau=120$.}

            Because the four potential events are all independent and
            disjoint, the probability of any one event is equal to the
            sum of the probabilities of each type of event. (Some might
            call this the addition rule of probability.) The resulting
            process is a poisson process with $\lambda = \sum_{i=0}^4
            \lambda_i$.

            Because the PMF of a Poisson is not invertible, we cannot
            sample M by applying the inverse function to a uniform
            random variable, as we did for generating random arrival
            times. A simple, though maybe not the most efficient,
            approach is to generate a table of the CDF, $F(.)$ and
            select the value M for which some $x \sim U \in [F(M),
            F(M+1))$.

            I'm not sure exactly how I would calculate the times $A_i$,
            such that $A_M$ is guaranteed to be within the given
            interval. That is, I could sample the inter-arrival times of
            the poisson distribution, just as I did in the previous
            version, but those random inter-arrival times could result
            in sum that is greater than T. That said, I'm not sure that
            I need to know the times $A_i$ for the sake of this
            simulation, because the number of bikes present at the
            station is not dependent on inter-arrival times.

            I can identify my events by noting that
            \[
                \Prob(\xi_1 \mid \{\xi_1,...,\xi_4\}) = \frac{\Prob(\xi_1)}
                {\Prob(\xi_1,...,\xi_4)}
            \]
            and create a CDF similarly to how I did for the value of M.

        \item \textbf{TO-DO: And this}
    \end{enumerate}

\end{solution}



\begin{problem} \marks{25}
There are $N$ individuals in a population, some of whom have a certain
viral infection that spreads as follows. Contacts between two members of
this population occur in accordance with a Poisson process of rate
$\la$. When a contact occurs, it is equally likely to involve any of the
$N\choose2$ pairs of individuals. If a contact involves an infected and
a healthy individual, then with probability $p$ the non-infected one
becomes infected. For Model 1 we assume that once infected, an
individual remains infected throughout. In Model 2, once infected, an
individual remains ill for an exponential amount of time with intensity
$\mu$, after which she/he becomes healthy again. We assume no deaths
occur, and there is no spontaneous infection. Let $X(t)$ denote the
number of infected individuals at time $t$. 
%For model 2 we assume that recovered individuals have life immunity so
%they never get the infection again. 
\begin{enumerate}
\item \marks{5} Show that $\{X(t), t\ge 0\}$ is a continuous time Markov
    Chain (CTMC). Specifically, show that it is a Birth and Death
    process.  For each of the models, specify the classes of the states
    and determine whether it is an absorbing Markov chain (identify the
    absorbing states) or an ergodic Markov Chain. Specify for each model
    the transition rates $q_{ij}$, the aggregate rate $v_1$ and the
    transition probabilities of the embedded chain $P_{ij}$. 

\item \marks{10} Starting with $i\ge1$ infected individuals, calculate
    the expected time until all members of the population are infected
    in Model 1. What is the expected time until absorption for Model 2?
    (you may use pseudocode to exhibit the solution, if not analytical).

\item \marks{5}  Explain how to do a simulation to estimate the time
    until absorption for the process of Model 2, starting at $X(0)=1$
    and perform the simulations. Include confidence intervals and
    explain how you calculate the variance. [{\it Hint:} Note that you
    have already calculated the expected value, which should help
    validate your simulation code]. 

\item \marks{5}  Explain how you would modify the model and the
    simulation code for the following scenario: contagion follows the
    same model as before, but once an individual is recovered, he/she
    has life immunity so they cannot get the infection again. 
\end{enumerate}
\end{problem}

\begin{solution}
    \begin{enumerate}
        \item Because infection is conditioned on the occurrence of an
            event in a Poisson process, and recovery occurs after an
            exponential time $\mu$, both infection and recovery occur
            in continuous time. Furthermore, because the rates infection
            and recovery are both functions of the current number of
            infected individuals, the process is wholly dependent on the
            current number of infected people, $X(t)$, meaning that
            $\{X(t), t \ge 0\}$ is a CTMC. At any given time, $X(t)$
            must either remain constant, or change by the addition of
            one infected individual (\emph{birth}) or the recovery of
            one infected individual (\emph{death}), meaning that the
            process is a \emph{birth and death process}.

            In $M_1$, individuals can never recover, meaning that $X(t)$
            is monotonically increasing. For this reason, no states
            communicate with each other, meaning that the process
            consists of $N + 1$ distinct classes, two of which are
            absorbing, $X(t) = 0$ and $X(t) = N$, and the rest of which
            are transient. $X(t) = 0$ is not an accessible state, but
            could be reached if $X(0) = 0$, in which case no further
            infections could occur. Transmission occurs when a contact
            occurs between one healthy and one infected individual.

            Using the law of total probability,
            \begin{align*}
                \Prob(\text{Infection}\mid\text{Contact})
                &=
                \Prob(\text{Transmission} \mid \text{MixedPair})
                \Prob(\text{MixedPair} \mid \text{Contact})\\
                &= pX(t)(N-X(t))\frac{2}{N(N - 1)}
            \end{align*}

            Meaning that the rate of infection is the rate of contact
            times the probability of infection, given contact:
            \[
                q_{i,i+1} = \frac{2\lambda p i(N-i)}{N(N-1)}
            \]

            In $M_1$, the aggregate rate is the same as $q$, because
            the probability of recovery is 0. If I understand the
            concept of the embedded Markov chain correctly, this is the
            state transition that occurs at every event, meaning that
            $P_{ij}=q_{ij} / \lambda$.
            \begin{align*}
                P_{i,i+1} &= \frac{2pi(N-i)}{N(N-1)}\\
                P_{ij} &= 0 \text{ for all other values of j}
            \end{align*}

            In $M_2$, $X(t)$ all states communicate except for $X(t) =
            0$, meaning that there are two classes, the absorbing class
            $\{X(t) = 0\}$ and the transient class, $\{0 < X(t) \le
            N\}$. The transmission (or \emph{birth}) rate remains the
            same, but there is now a recovery (or \emph{death}) rate of
            $\mu X(t)$. This makes the aggregate rate:
            \[
                v_i = \frac{2\lambda p i(N-i)}{N(N-1)} + \mu i
            \]

            \begin{align*}
                P_{i,i+1} &= \Prob(\text{Infection} \mid \text{Contact},i)
                \Prob(\text{Contact})
                (1 - \Prob(\text{Recovery}\mid i))\\
                P_{i,i} &= \Prob(\text{Infection} \mid \text{Contact}, i)
                \Prob(\text{Contact})\Prob(\text{Recovery}\mid i)\\
                P_{i,i-1} &= (1- \Prob(\text{Infection} \mid
                \text{Contact},i))\Prob(\text{Recovery}\mid i)
            \end{align*}

        \item
            [Is using psuedocode to exhibit the solution different from
            using psuedocode to simulate? This is unclear.]

            We can use discrete event simulation for both $M_1$ and
            $M_2$.

            \begin{python}
def simulate_m1(total_pop, initial_infected):
    t = 0
    state = {'total': total pop, 'infected': initial_infected}
    future_events = PriorityQueue()
    future_events.put(Contact()) # initializes contact with 
                                 # arrival time per Poisson dist.
    while state['infected'] != state['total']:
        event = future_events.get() # get soonest event
        event.trigger(state)     # transmit illness per P_{ij}
        future_events.put(event.get_next()) # queue next contact
        t = event.time
    return t
\end{python}

            We can adapt this for $M_2$ by adding a recovery event for
            each contact that results in an infection. (This is not the
            most efficient solution to simulate, but it is the most
            efficient solution to code!)

            \begin{python}
def simulate_m2(total_pop, initial_infected):
    t = 0
    state = {'total': total pop, 'infected': initial_infected}
    future_events = PriorityQueue()
    future_events.put(Contact()) # initializes contact with 
                                 # arrival time per Poisson dist.
    while state['infected'] != state['total']:
        event = future_events.get() # get soonest event
        if event.trigger(state): 
            # Contact.trigger() returns True if infection occurs
            future_events.put(Recovery(t)) # add recovery event
        # transmit illness per P_{ij}
        future_events.put(event.get_next()) # queue next contact
        t = event.time
    return t
            \end{python}

        \item See above for a simulation for $M_2$. I'm not sure what
            values I am to report, as the outcome is dependent on the
            size of the population.

        \item The simplest, though possibly not most efficient way,
            would be to add an additional rv \emph{Recovered}, which
            would track the number of individuals who have recovered.
            Now,
            \[
                \Prob(\text{MixedPair} \mid \text{Contact}, i,
                \text{Recovered}) = 
                \frac{2i(N - i - \text{Recovered})}{N(N-1)}
            \]
            In other words, possible pairs between infected individuals
            and healthy individuals who have not yet recovered, over
            total possible pairs.

    \end{enumerate}
\end{solution}


\begin{problem} \marks{30}
You have decided to do consultation for modeling, simulation and
optimization. You offer various research  services: modeling,
statistical analysis, optimization, software development, etc. There are
$N$ such research stages (or ``tasks'') and they always follow a
specific order: research of type  $n$ is always followed  by that of
type $n-1$, for $N\ge n > 1$. The time (in hours) required to complete
stage $n$ follows a distribution $F_n$ of mean $\mu_n$. Potential
clients arrive according to a Poisson process of rate $\la$, and you
take the job only if you are free at the time of arrival of the client.
If you are already working on a problem then you do not take new
contracts. Each problem starts at stage $n$ with probability $p_n$ and
ends at stage $1$, following all intermediate stages. Let $X(t)=n$ if at
time $t$ you are working on stage $n$ of a problem, and use $X(t)=0$ if
at time $t$ you are free, waiting for new contracts. 
\begin{enumerate}
\item \marks{5} Under what conditions on $\{F_n\}$ is $X(t)$ a CTMC?
    Give the rates $v_i$ and transition kernel $P_{ij}$.
\item \marks{10} Suppose that the distributions $F_n, n=1,\ldots , N$
    are not memoryless. You assume that you will charge $c$ dollars per
    hour of work. Specify the regeneration points of the process (see
    CTMC Lecture Notes) and determine the long term rate of profit. 
\item \marks{15} In order to price your services correctly, let us
    assume that clients are discouraged if $c$ is too big. Specifically,
    assume that the probability that clients accept your conditions is
    given by $P_c=(K-c)/K, 0\le c <K$. If $c\ge K$ then $P_c=0$ (no
    client will hire you). %and define 
%\[
%M\eqdef \sum_{n=1}^N p_n \sum_{k=n}^N \mu_k.
%\]
Design a simulation to evaluate the long term profit as a function of
$c$. Explain how you can use these simulations to determine the optimal
value of $c$. Specify how you build the simulations for different
values of $c$ in order to compare the long term profit rate, and how do
you build the confidence intervals. Plus, specify if you are using a
linear or binary search to determine the optimal value. For you
simulation, use $K=2, \la=2$, $N=5$ and 
\[
    \mu= (0.3,  0.1, 0.2, 0.3, 0.1)^\top.
\]

\end{enumerate}

\end{problem}

\begin{solution}

    \textbf{TO-DO}

\end{solution}


\end{document}

