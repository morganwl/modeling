\documentclass[12pt]{article}

\usepackage{../modelset}


\title{\bfseries Computer Modeling  and  Simulation}
\author{Morgan Wajda-Levie}
\date{November 6, 2022}


\begin{document}\sf

\lhead{ } \chead{ } \rhead{ } \lfoot{ } \cfoot{\sf -- page
\thepage~  of  \pageref{LastPage}  --  } \rfoot{ }                                                     %            <=====   PAGE NUMBERING

\maketitle

%\null \vskip - 5 cm

\begin{center}
\vskip - 1.5 em { {\bfseries Problem Set 3}}
\end{center}

\begin{problem}\marks{15}
    Let $P^{(i)}$ be the transition matrix of an ergodic Markov chain
    with limit probabilities $\pi^{(i)}$, for $i=1,2$. 
    \begin{enumerate}
        \item \marks{5} Let $X_0=1$ be given. If the result of a coin
            toss is Heads (H) then $\{X_n\}$ will follow the dynamics
            given by $P^{(1)}$. Otherwise it will follow those given by
            $P^{(2)}$. Is $\{X_n\}$ a Markov chain? If so, determine the
            transition probabilities. Letting  $p=\Prob(H)$, calculate
            the limiting probabilities: $\lim_{n\to\infty}
            \Prob(X_n=i)$?

        \item \marks{10} Now suppose that at each step we first throw a
            coin. If it results $H$ then the following state is chosen
            according to $P^{(1)}$, otherwise it is chosen according to
            $P^{(2)}$. Is $\{X_n\}$ a Markov chain? If so, determine the
            transition probabilities. Show  by counterexample that the
            limit probabilities are not the same as in (a) above.
    \end{enumerate}
\end{problem}

\begin{solution}

    Assume that there exists some states $i,j$ where $j$ is accessible
    from $i$ and the minimum number of steps to reach $j$ from $i$ is
    $N$, where $N > |S|$.

    By the Markov property, this means that there exists some $X_0,
    X_1,...,X_T$ of length $N$ such that $X_0 = i, X_T=j$ and $\Prob(X_n
    \mid X{n-1}) > 0$ and $\Prob(X_n \mid X_m) = 0$ for all $m < n - 1$,
    because, if there exists a path from $X_m$ to $X_n$, then $X_T$ can
    be reached without passing through $X_{m+1},\ldots,X_{n-1}$,
    resulting in a sequence of length $<N$.

    However, if $N > |S|$, there must exist some distinct $X_n,X_m$ such
    that $X_n=X_m$, meaning that $\Prob(X_n \mid X_{m-1}) > 0$ and $m -
    1 < n -1$, which contradicts the Markov property.

    \raggedleft $\square$

\end{solution}


\begin{problem} \marks{10}
Consider a population of individuals each of whom possesses two genes
that can be either type $A$ or type $a$. Suppose that in outward
appearance (the {\em phenotype}) type $A$ is dominant and type $a$
recessive. Suppose that the population has stabilized, and the
percentages of individuals having respective gene pairs $AA$, $aa$ and
$Aa$ are $p,q,$ and $r$ Call an individual dominant or recessive
depending on phenotype. Let $S_{11}$ denote the probability that an
offspring of two dominant parents will be recessive, $S_{10}$ the
probability that the offspring of one recessive and one dominant parent
will be recessive. Compute $S_{11}$ and $S_{10}$ to show that $S_{11}=
S_{10}^2$. These quantities are known in genetics as {\em Snyder's
ratios}. 
\end{problem}

\begin{solution}
    % type A: dominant
    % type a: recessive
    % p = probability of AA in population
    % q = probability of aa in population
    % r = probability of Aa in population

    Let $d$ be the probability that a single parent with a dominant
    phenotype will pass a recessive gene to their children. Because a
    parent with a recessive phenotype will pass a recessive gene on to
    their child w.p.1., $S_{10} = d$. Since $S_{11} = d * d$, we can
    show that $S_{11} = S_{10}$ without even calculating the probability
    for $d$.

    Because $AA$ is unable to transmit a recessive gene, and assuming
    the selection of genes to be transmitted is uniform, $d$ can be
    calculated as the probability of transmitting $a$ given $Aa$ times
    the probability of $Aa$ given $AA$ or $Aa$,
    $d = \frac{r}{2(p + r)}$.

    \begin{align*}
        \Prob(S_{11}) &= \left(\frac{r}{2(p+r)}\right)^2 \\
        \Prob(S_{10}) &= \frac{r}{2(p+r)}
    \end{align*}

\end{solution}


\begin{problem}\marks{20}
An algorithm is built to find the zeroes of a function. If this
algorithm is in state $j$ at the $n$-th step, then the probability of
finding a zero in the next step is $1/j$. Otherwise its state will be $k
\in \{1,2,\ldots, j-1\}$ with probability $2k^/j^2$. 
\begin{enumerate}
\item \marks{10} Find the mean number of iterations of the algorithm
    when we start at state $m$ and show that this is $O(\log_2 m)$. 
\item \marks{10} Simulate this process as a Markov chain and estimate
    the number of iterations until the algorithm finds the zero. Explain
    how you write the code  in order to achieve a  relative error of
    $5\%$ in your final estimate.  Verify that your simulation results
    are consistent with your theoretical answer above. 
\end{enumerate}
\end{problem}

\begin{solution}
    \begin{enumerate}
        \item
            Because the state following any state $j$ is always chosen
            from states in $\{0,1,\ldots,j-1\}$, we can see that this
            process will always monotonically decrease towards $0$. We
            can see that, if a given state $k$ is stated as a fraction
            of the state $j$ that proceeds it, that
            \[
                m \prod_{n=0}^t \frac{k}{X_n}
            \]
            will shrink towards $0$ as $t$ approaches $\tau$, where
            $\tau$ is the total number of steps until the process is
            absorbed in state $0$.

            We can calculate an expected value of $k$, given $j$, by
            summing the possible values of $k$, multiplied by their
            respective probability, given $j$.
            \begin{align*}
                \frac{2}{j^2} \sum_{k=1}^{j-1}k^2
                &= \frac{2(j-1)(j)(2j)}{6j^2}\\
                &= \frac{2j^3-2j^2}{3j^2}\\
                &= \frac{2j-2}{3} \\
                \mathbb E[k \mid j] &= \frac{2j-2}{3}
            \end{align*}

            Substituting $k$ for its expected value, we can see that,
            given sufficiently large $m$:

            \begin{align*}
                1 &= \left\lceil m\prod_{n=0}^\tau \frac{2 X_n-2}{3 X_n}
                \right\rceil\\
                  & \approx  m \prod_{n=0}^\tau \frac{2}{3}\\
                  &= m \left(\frac{2}{3}\right)^\tau\\
                  \text{such that}\\
                    \mathbb E[\tau] &=2\log_3m
            \end{align*}

            Because $O()$ represents an upper bound, this value is also
            $O(log_2m)$.

        \item

            The simulation starts with a value $m$ (the initial state),
            and continuously samples a new value $k$ based on the
            current state, until $k = 0$. To generate the random
            variable $k$ according to the given probabilities, a reverse
            linear search is used, starting with the largest and most
            probable value of $k$, and decreasing $k$ until the value
            according to $1 - F(k)$. Moving backwards improves
            performance over a forward search by a factor of $\approx
            2$.

            In order to guarantee no more than a 5\% error rate, I
            performed the simulation in batches, gathering the mean
            simulated value and the sample variance. After each batch,
            or \emph{replication}, I calculated the running average for
            the variance, and divided this by the total simulations.
            Once this value was within my desired confidence ($<0.05$),
            I returned the observed mean and variance.

            While I could not calculate the mean iterations required for
            an arbitrary $m$, I tested a number of large values of $m$,
            and found that the theoretical minus the observed was always
            less than 1 (and greater than 0) and generally less than
            0.5, so I would say that this is reasonably consistent.

            \newpage
            \begin{python}

def sample_k(j):
    """Generate a value for k, given j, according to specified
    probabilities."""
    u = random()
    k = j - 1
    cdf = 2 * k / j**2
    while k > 0:
        if u <= cdf:
            return k
        k -= 1
        cdf += 2 * k / j**2
    return 0
            \end{python}

    \end{enumerate}
\end{solution}



\begin{problem} \marks{25}
There are $N$ individuals in a population, some of whom have a certain
viral infection that spreads as follows. Contacts between two members of
this population occur in accordance with a Poisson process of rate
$\la$. When a contact occurs, it is equally likely to involve any of the
$N\choose2$ pairs of individuals. If a contact involves an infected and
a healthy individual, then with probability $p$ the non-infected one
becomes infected. For Model 1 we assume that once infected, an
individual remains infected throughout. In Model 2, once infected, an
individual remains ill for an exponential amount of time with intensity
$\mu$, after which she/he becomes healthy again. We assume no deaths
occur, and there is no spontaneous infection. Let $X(t)$ denote the
number of infected individuals at time $t$. 
%For model 2 we assume that recovered individuals have life immunity so
%they never get the infection again. 
\begin{enumerate}
\item \marks{5} Show that $\{X(t), t\ge 0\}$ is a continuous time Markov
    Chain (CTMC). Specifically, show that it is a Birth and Death
    process.  For each of the models, specify the classes of the states
    and determine whether it is an absorbing Markov chain (identify the
    absorbing states) or an ergodic Markov Chain. Specify for each model
    the transition rates $q_{ij}$, the aggregate rate $v_1$ and the
    transition probabilities of the embedded chain $P_{ij}$. 

\item \marks{10} Starting with $i\ge1$ infected individuals, calculate
    the expected time until all members of the population are infected
    in Model 1. What is the expected time until absorption for Model 2?
    (you may use pseudocode to exhibit the solution, if not analytical).

\item \marks{5}  Explain how to do a simulation to estimate the time
    until absorption for the process of Model 2, starting at $X(0)=1$
    and perform the simulations. Include confidence intervals and
    explain how you calculate the variance. [{\it Hint:} Note that you
    have already calculated the expected value, which should help
    validate your simulation code]. 

\item \marks{5}  Explain how you would modify the model and the
    simulation code for the following scenario: contagion follows the
    same model as before, but once an individual is recovered, he/she
    has life immunity so they cannot get the infection again. 
\end{enumerate}
\end{problem}

\begin{solution}
    
    \textbf{TO-DO}

\end{solution}


\begin{problem} \marks{30}
You have decided to do consultation for modeling, simulation and
optimization. You offer various research  services: modeling,
statistical analysis, optimization, software development, etc. There are
$N$ such research stages (or ``tasks'') and they always follow a
specific order: research of type  $n$ is always followed  by that of
type $n-1$, for $N\ge n > 1$. The time (in hours) required to complete
stage $n$ follows a distribution $F_n$ of mean $\mu_n$. Potential
clients arrive according to a Poisson process of rate $\la$, and you
take the job only if you are free at the time of arrival of the client.
If you are already working on a problem then you do not take new
contracts. Each problem starts at stage $n$ with probability $p_n$ and
ends at stage $1$, following all intermediate stages. Let $X(t)=n$ if at
time $t$ you are working on stage $n$ of a problem, and use $X(t)=0$ if
at time $t$ you are free, waiting for new contracts. 
\begin{enumerate}
\item \marks{5} Under what conditions on $\{F_n\}$ is $X(t)$ a CTMC?
    Give the rates $v_i$ and transition kernel $P_{ij}$.
\item \marks{10} Suppose that the distributions $F_n, n=1,\ldots , N$
    are not memoryless. You assume that you will charge $c$ dollars per
    hour of work. Specify the regeneration points of the process (see
    CTMC Lecture Notes) and determine the long term rate of profit. 
\item \marks{15} In order to price your services correctly, let us
    assume that clients are discouraged if $c$ is too big. Specifically,
    assume that the probability that clients accept your conditions is
    given by $P_c=(K-c)/K, 0\le c <K$. If $c\ge K$ then $P_c=0$ (no
    client will hire you). %and define 
%\[
%M\eqdef \sum_{n=1}^N p_n \sum_{k=n}^N \mu_k.
%\]
Design a simulation to evaluate the long term profit as a function of
$c$. Explain how you can use these simulations to determine the optimal
value of $c$. Specify how you build the simulations for different
values of $c$ in order to compare the long term profit rate, and how do
you build the confidence intervals. Plus, specify if you are using a
linear or binary search to determine the optimal value. For you
simulation, use $K=2, \la=2$, $N=5$ and 
\[
    \mu= (0.3,  0.1, 0.2, 0.3, 0.1)^\top.
\]

\end{enumerate}

\end{problem}

\begin{solution}

    \textbf{TO-DO}

\end{solution}


\end{document}

