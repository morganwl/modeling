\begin{solution}
    \begin{enumerate}
        \item Because infection is conditioned on the occurrence of an
            event in a Poisson process, and recovery occurs after an
            exponential time $\mu$, both infection and recovery occur
            in continuous time. Furthermore, because the rates infection
            and recovery are both functions of the current number of
            infected individuals, the process is wholly dependent on the
            current number of infected people, $X(t)$, meaning that
            $\{X(t), t \ge 0\}$ is a CTMC. At any given time, $X(t)$
            must either remain constant, or change by the addition of
            one infected individual (\emph{birth}) or the recovery of
            one infected individual (\emph{death}), meaning that the
            process is a \emph{birth and death process}.

            In $M_1$, individuals can never recover, meaning that $X(t)$
            is monotonically increasing. For this reason, no states
            communicate with each other, meaning that the process
            consists of $N + 1$ distinct classes, two of which are
            absorbing, $X(t) = 0$ and $X(t) = N$, and the rest of which
            are transient. $X(t) = 0$ is not an accessible state, but
            could be reached if $X(0) = 0$, in which case no further
            infections could occur. Transmission occurs when a contact
            occurs between one healthy and one infected individual.

            Using the law of total probability,
            \begin{align*}
                \Prob(\text{Infection}\mid\text{Contact})
                &=
                \Prob(\text{Transmission} \mid \text{MixedPair})
                \Prob(\text{MixedPair} \mid \text{Contact})\\
                &= pX(t)(N-X(t))\frac{2}{N(N - 1)}
            \end{align*}

            Meaning that the rate of infection is the rate of contact
            times the probability of infection, given contact:
            \[
                q_{i,i+1} = \frac{2\lambda p i(N-i)}{N(N-1)}
            \]

            In $M_1$, the aggregate rate is the same as $q$, because
            the probability of recovery is 0. If I understand the
            concept of the embedded Markov chain correctly, this is the
            state transition that occurs at every event, meaning that
            $P_{ij}=q_{ij} / \lambda$.
            \begin{align*}
                P_{i,i+1} &= \frac{2pi(N-i)}{N(N-1)}\\
                P_{ij} &= 0 \text{ for all other values of j}
            \end{align*}

            In $M_2$, $X(t)$ all states communicate except for $X(t) =
            0$, meaning that there are two classes, the absorbing class
            $\{X(t) = 0\}$ and the transient class, $\{0 < X(t) \le
            N\}$. The transmission (or \emph{birth}) rate remains the
            same, but there is now a recovery (or \emph{death}) rate of
            $\mu X(t)$. This makes the aggregate rate:
            \[
                v_i = \frac{2\lambda p i(N-i)}{N(N-1)} + \mu i
            \]

            \begin{align*}
                P_{i,i+1} &= \Prob(\text{Infection} \mid \text{Contact},i)
                \Prob(\text{Contact})
                (1 - \Prob(\text{Recovery}\mid i))\\
                P_{i,i} &= \Prob(\text{Infection} \mid \text{Contact}, i)
                \Prob(\text{Contact})\Prob(\text{Recovery}\mid i)\\
                P_{i,i-1} &= (1- \Prob(\text{Infection} \mid
                \text{Contact},i))\Prob(\text{Recovery}\mid i)
            \end{align*}

        \item
            [Is using psuedocode to exhibit the solution different from
            using psuedocode to simulate? This is unclear.]

            We can use discrete event simulation for both $M_1$ and
            $M_2$.

            \begin{python}
def simulate_m1(total_pop, initial_infected):
    t = 0
    state = {'total': total pop, 'infected': initial_infected}
    future_events = PriorityQueue()
    future_events.put(Contact()) # initializes contact with 
                                 # arrival time per Poisson dist.
    while state['infected'] != state['total']:
        event = future_events.get() # get soonest event
        event.trigger(state)     # transmit illness per P_{ij}
        future_events.put(event.get_next()) # queue next contact
        t = event.time
    return t
\end{python}

            We can adapt this for $M_2$ by adding a recovery event for
            each contact that results in an infection. (This is not the
            most efficient solution to simulate, but it is the most
            efficient solution to code!)

            \begin{python}
def simulate_m2(total_pop, initial_infected):
    t = 0
    state = {'total': total pop, 'infected': initial_infected}
    future_events = PriorityQueue()
    future_events.put(Contact()) # initializes contact with 
                                 # arrival time per Poisson dist.
    while state['infected'] != state['total']:
        event = future_events.get() # get soonest event
        if event.trigger(state): 
            # Contact.trigger() returns True if infection occurs
            future_events.put(Recovery(t)) # add recovery event
        # transmit illness per P_{ij}
        future_events.put(event.get_next()) # queue next contact
        t = event.time
    return t
            \end{python}

        \item See above for a simulation for $M_2$. I'm not sure what
            values I am to report, as the outcome is dependent on the
            size of the population.

        \item The simplest, though possibly not most efficient way,
            would be to add an additional rv \emph{Recovered}, which
            would track the number of individuals who have recovered.
            Now,
            \[
                \Prob(\text{MixedPair} \mid \text{Contact}, i,
                \text{Recovered}) = 
                \frac{2i(N - i - \text{Recovered})}{N(N-1)}
            \]
            In other words, possible pairs between infected individuals
            and healthy individuals who have not yet recovered, over
            total possible pairs.

    \end{enumerate}
\end{solution}
