\question{3}
We observe that the events ${A_1,...,A_13}$ are not conditionally
independent, by considering the case where a $2$ has been dealt first,
meaning that $P{A_2|D_1=2}=0$. Likewise, we can observe that
$P(A_13|A_1,...,A_12)=1$. Because the events are not conditionally
independent, this is not a true binomial distribution; because the
outcome of each event is determined by the way the deck has been
shuffled, we can consider this an \emph{ordering problem.}

Because I am unsure of how to model this distribution, I chose to
simulate 100,000,000 deals and estimate the probability
distribution through observation.

\begin{python}
def simulate_matches():
  deck = list(range(13)) # create an ordered deck of 13 cards
  shuffle(deck)
  return sum([i == card for i, card in enumerate(deck)])

count = [0]*14
samples = int(1e8)
for i in range(samples):
  count[simulate_matches() += 1]
expect = sum([i*c for i,c in enumerate(count)])/samples
\end{python}

This procedure estimates $E[N] = 1.000$.
