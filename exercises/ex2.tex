\documentclass{article}
\usepackage{../modeling}

\title{Chapter 2: Introduction to Simulation and Generation of Random
Variables}

\begin{document}

\begin{enumerate}

    \question{1} Think about a transportation problem. Describe the
    problem for simulation. Describe the target process. Include the
    questions to be addressed.

    I would like to determine the impact of fare collection on bus
    performance.

    The target process is that of the passengers in the system (waiting
    for a bus or riding a bus) and the position of each bus on their
    route. At any time, passengers can arrive to wait for a bus. Buses
    can either load and unload passengers (when they have arrived at a
    stop), or move through their route.

    Random variables are number of passengers waiting at a stop, number
    of passengers exiting at each stop, loading and unloading time of
    passengers (with and without fare collection), and travel time
    between stops.

    Questions are how much the expected travel time for passengers
    changes when removing fare collection from the system. Does the
    elimination of fare collection allow the same quality of service
    while running fewer buses?

    \question{2} 

    Assuming a discrete time process, $\Omega$ could be composed of the
    passenger arrival event, the passenger stop event, the loading
    progress event and the bus travel progress event.

    \question{3}

    \begin{enumerate}
        \item $P\{x\in\{1,2,3\} \mid x < 4\} = \frac{1}{3}$. Because we
            only consider instances of $x < 4$, the total probability of
            this event (which happens to be $\frac{1}{2}$), is
            insignificant, so long as it is non-zero.
        \item $P\{x\in\{1,2\} = P\{x = 1\} + P\{x = 2\}$, because these
                are disjoint events. As such, the probability of an
                outcome that is one of two numbers is $\frac{1}{3}$.
        \item The former method has the potential for requiring multiple
            rolls to achieve sample, but requires very little meaningful
            computation to interpret the results, which could be
            meaningful for a human. The latter method allows one sample
            to be made with one roll, but requires some computation to
            interpet the results.
    \end{enumerate}

\end{enumerate}

\end{document}
