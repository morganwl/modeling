\documentclass[xcolor=x11names,compress, 10pt]{beamer}
\usepackage{graphicx}
\usepackage{tikz}
\usetikzlibrary{decorations.fractals}
%% Beamer Layout %%%%%%%%%%%%%%%%%%%%%%%%%%%%%%%%%%
\useoutertheme[subsection=false,shadow]{miniframes}
\useinnertheme{default}
%\usefonttheme{serif}
%\usepackage{palatino}

% If small caps desired:
%\setbeamerfont{title like}{shape=\scshape}
%\setbeamerfont{frametitle}{shape=\scshape}

\setbeamertemplate{footline}[frame number] % for page numbers


\setbeamerfont{title like}{shape=\bfseries}
\setbeamerfont{frametitle}{shape=\bfseries}
\setbeamerfont{alerted text}{series=\it}
\setbeamerfont{block title}{shape=\bfseries}
\setbeamercolor{block title}{fg=orange}
\setbeamercolor*{lower separation line head}{bg=orange} 
\setbeamercolor*{normal text}{fg=black,bg=white} 
\setbeamercolor*{alerted text}{fg=orange} 
\setbeamercolor*{example text}{fg=black} 
\setbeamercolor*{structure}{fg=black} 
 
\setbeamercolor*{palette tertiary}{fg=black,bg=black!10} 
\setbeamercolor*{palette quaternary}{fg=black,bg=black!10} 

\renewcommand{\(}{\begin{columns}}
\renewcommand{\)}{\end{columns}}
\newcommand{\<}[1]{\begin{column}{#1}}
\renewcommand{\>}{\end{column}}
%%%%%%%%%%%%%%%%%%%%%%%%%%%%%%%%%%%%%%%%%%%%%%%%%%

\usepackage[noend]{algpseudocode}

\begin{document}

\begin{frame}
\title{\bfseries Improvements in Bus Travel Times by Eliminating Fare
Collection}
\author{Morgan Wajda-Levie}


\titlepage
\end{frame}

\section{Motivation}
\begin{frame}
\frametitle{Motivation}

In 2019, the MTA's bus system served an average of 1.8 million riders
per weekday in New York City. Shorter bus travel times not only improve
the lives of nearly 2 million riders, but could encourage more riders to
favor bus travel over private automobiles, resulting in reduced traffic
and pollution for all New Yorkers.

\end{frame}

\begin{frame}
    \frametitle{Eliminating fare collection}

    While buses have multiple doors for disembarking passengers, all
    passengers must board through the front door, where they pay a fare.
    This fare collection imposes a bottleneck, particularly for buses on
    high-traffic routes.

    \medskip

    \textbf{How would passenger travel times change if fare collection
    were eliminated?}

\end{frame}

\section{Methodology}
\begin{frame}
\frametitle{Model}

    At any given time, a bus route has the following
    characteristics:
    \begin{itemize}
        \item Position of each bus along route
        \item Number of passengers on each bus traveling to each future
            stop along route
        \item Number of passengers queued at each stop
        \item Traffic conditions
    \end{itemize}

    This can be considered a \emph{continuous time Markov chain},
    modeled as a \emph{discrete event simulation}. This is a stationary
    problem, because we will be looking at the aggregate change in travel
    time continuous bus service.
\end{frame}

\begin{frame}
    \frametitle{Random variables}

    \begin{itemize}
        \item $Waiting(s, t)$: Count of passengers at stop $s$ at time
            $t$.
        \item $Destinations(b, s, t)$: Count of passengers on bus
            $b$ headed to stop $s$. A passenger's destination is assumed
            to be independent of a passenger's origin. Based on the rate
            $destinations_s$, which should sum to be equal to
            $departures$.
        \item $BoardingTime(s, t)$: The combined loading and unloading
            time of a bus at stop $s$ at time $t$, conditioned on the
            number of passengers entering and exiting the bus, a loading
            rate $r_{l,s,t}\sim\mathcal N$ and unloading rate
            $r_{u,s,t}\sim\mathcal N$.
        \item $TravelTime(s, d, t)$: The travel time from stop $s$ to
            stop $d$, conditioned on a deterministic distance and
            $t_{s,d,t}\sim\mathcal N$, the traffic between $s$ and $d$.
    \end{itemize}
\end{frame}

\begin{frame}
    \frametitle{Waiting passengers}

    \[
        Waiting(s, t) \sim \text{Poisson}(\lambda^{(w)}_s, t -
        t_{0,s})
    \]

    Each bus stop is an independent Poisson counting process which grows
    from the moment a bus departs to the moment another bus arrives.
    This value can be measured for every bus arrival, but must also be
    measured again before the bus departs.

\end{frame}

\begin{frame}
    \frametitle{Passenger destinations}

    At boarding time, the number of passengers headed for each
    destination is a multinomial distribution, where $P(D = s) =
    \frac{\lambda^{(d)}_s}{\sum_{i=s}^n\lambda^{(d)}_i}$.

    At any stop $s$, this value can be measured as a binomial random
    variable, conditioned on the number of passengers currently on the
    bus.

\end{frame}

\begin{frame}
    \frametitle{Simulation Models}

    I will be using Discrete Event Simulation.

    \begin{itemize}
        \item \textbf{Fare collection}: Large $r_{l,s,t}$ to account for
            boarding through only one door and the time of fare
            collection.
        \item \textbf{Fareless}: $r_{l,s,t} = r_{u,s,t}$, a smaller
            value to account for passengers boarding through all
            available doors, without any time for fare collection.
    \end{itemize}
\end{frame}

\begin{frame}
    \frametitle{Measured outcomes}

    My goal is to measure the affect on cumulative mean travel times if
    fare collection is eliminated. This is equal to the sum of time
    between events for any bus $b$, times the number of passengers on
    bus $b$.

\end{frame}


\begin{frame}
\frametitle{Preliminary results}

I have constructed a prototype simulation model, but have not yet
constructed a route to provide meaningful results.
\end{frame}

\section{Discussion} 

\begin{frame}
\frametitle{Possible extensions}

\begin{itemize}
    \item Traffic lights (not present for all stops)
    \item Bus schedules
    \item Connections between buses and other transit
    \item Advanced traffic modeling
    \item 24-hour performance with time-dependent probabilities
\end{itemize}

\end{frame}

\begin{frame}
    \frametitle{Timetable}

    \begin{tabular}{ll}
        11-28 & Finish simulation code \\
        12-5 & Finish constructing route model \\
        12-12 & Acquire results \\
        12-19 & Deliver report \\
    \end{tabular}
\end{frame}


\end{document}




